% Title:   UofT Art & Sciences Assignment Sample File
% Version: 1.00
% Author:  Kaveh Ghasemloo
% Date:    Sept. 28, 2012
%
% Licence: 
% This work is licensed under the Creative Commons Attribution-ShareAlike 3.0 Unported License. To view a copy of this license, visit http://creativecommons.org/licenses/by-sa/3.0/ or send a letter to Creative Commons, 444 Castro Street, Suite 900, Mountain View, California, 94041, USA.

\documentclass[10pt]{csc_assignment}
\usepackage[]{algorithm2e}
\usepackage{amsmath}
\usepackage{algpseudocode}
\usepackage{qtree}

% ----------------------------------------------------------------
% TODO: Enter the assignment number, your name, and your student number below
% ----------------------------------------------------------------
\AssignmentName{3}
\QuestionCount{8}
\StudentName{John Armstrong, Henry Ku}
\StudentNumber{993114492\textbackslash g2jarmst, 998551348\textbackslash g2kuhenr}

% ----------------------------------------------------------------
\begin{document}


\Acknowledgements{
% ----------------------------------------------------------------
% TODO: Write your acknowledgements below.
% ----------------------------------------------------------------

"We declare that we have not used any outside help in completing this assignment."

% ----------------------------------------------------------------
% Aacknowledgements ends
% ----------------------------------------------------------------
}
\begin{description}

\newpage
\item[Q1.]
% ----------------------------------------------------------------
% TODO: Write your answer to the question below. 
% ----------------------------------------------------------------

Suppose G(V, E) is representative of a network where V = \{s, t, v$_{1}$, ..., v$_{n}$\} 
and E is the edges that make up the network. Suppose we have sets S and T such that
s $\in$ S, t $\in$ T, (S $\cup$ T) - \{s, t\} = V, and S $\cap$ T = $\emptyset$. Also, 
suppose that for all possible arrangement of vertices in S and T, c(S, T) = f(S, T), which means that G(V, E) has an exponential number of minimum cuts between s and t.\\
To construct a graph that is representative of there being 2$^{n}$ possible minimum cuts, a tree
like structure seems appropriate since there are 2$^{n}$ leaves in a full tree with n nodes. Taking inspiration from Huffman encoding, we wish to have a graph that for each leaf in a full tree will represent a unique binary string that represents whether a vertex is in S or T. Suppose we arrange the the string as follows, v$_{1}$, ..., v$_{n}$, where if v$_{i}$ is in S then v$_{i}$ = 0
and v$_{i}$ = 1 if it is in T.

% ----------------------------------------------------------------
% Answer ends
% ----------------------------------------------------------------

\newpage
\item[Q2.]
% ----------------------------------------------------------------
% TODO: Write your answer to the question below. 
% ----------------------------------------------------------------

% ----------------------------------------------------------------
% Answer ends
% ----------------------------------------------------------------

\newpage
\item[Q3.]
% ----------------------------------------------------------------
% TODO: Write your answer to the question below. 
% ----------------------------------------------------------------

% ----------------------------------------------------------------
% Answer ends
% ----------------------------------------------------------------

\newpage
\item[Q4.]
% ----------------------------------------------------------------
% TODO: Write your answer to the question below. 
% ----------------------------------------------------------------
    
% ----------------------------------------------------------------
% Answer ends
% ----------------------------------------------------------------

\newpage
\item[Q5.]
% ----------------------------------------------------------------                                                                               
% TODO: Write your answer to the question below.                                                                                                 
% ----------------------------------------------------------------                                                                               

% ----------------------------------------------------------------                                                                               
% Answer ends                                                                                                                                    
% ---------------------------------------------------------------- 


\newpage
\item[Q6.]
% ----------------------------------------------------------------                                                                               
% TODO: Write your answer to the question below.                                                                                                 
% ----------------------------------------------------------------                                                                               

% ----------------------------------------------------------------                                                                               
% Answer ends                                                                                                                                    
% ----------------------------------------------------------------


\newpage
\item[Q7.]
% ----------------------------------------------------------------                                                                               
% TODO: Write your answer to the question below.                                                                                                 
% ----------------------------------------------------------------                                                                               

% ----------------------------------------------------------------                                                                               
% Answer ends                                                                                                                                    
% ----------------------------------------------------------------


\newpage
\item[Q8.]
% ----------------------------------------------------------------                                                                               
% TODO: Write your answer to the question below.                                                                                                 
% ----------------------------------------------------------------                                                                               

% ----------------------------------------------------------------                                                                               
% Answer ends                                                                                                                                    
% ----------------------------------------------------------------
\end{description}
\end{document}
